%%
%% Dies ist die Datei \u201esample-manuscript.tex\u201d,
%% die mit dem Dienstprogramm docstrip erstellt wurde.
%%
%% Die urspr�nglichen Quelldateien waren:
%%
%% samples.dtx  (mit Optionen: \u201eall,proceedings,bibtex,manuscript\u201d)
%% 
%% WICHTIGER HINWEIS:
%% 
%% Das Copyright finden Sie in der Quelldatei.
%% 
%% Alle modifizierten Versionen dieser Datei m�ssen umbenannt werden
%% und einen neuen Dateinamen erhalten, der sich von \u201esample-manuscript.tex\u201d unterscheidet.
%% 
%% Informationen zur Verbreitung der Originalquelle finden Sie in den Bedingungen
%% f�r das Kopieren und �ndern in der Datei samples.dtx.
%% 
%% Diese generierte Datei darf verbreitet werden, solange die
%% oben aufgef�hrten Originalquelldateien Teil derselben
%% Verbreitung sind. (Die Quellen m�ssen sich nicht unbedingt
%% im selben Archiv oder Verzeichnis befinden.)
%%
%%
%% Befehle f�r TeXCount
%TC:macro \cite [Option:Text,Text]
%TC:macro \citep [Option:Text,Text]
%TC:macro \citet [Option:Text,Text]
%TC:envir table 0 1
%TC:envir table* 0 1
%TC:envir tabular [ignore] Wort
%TC:envir displaymath 0 Wort
%TC:envir math 0 Wort
%TC:envir comment 0 0
%%
%% Der erste Befehl in Ihrer LaTeX-Quelle muss der Befehl \documentclass
%% sein.
%%
%% F�r die Einreichung und �berpr�fung Ihres Manuskripts �ndern Sie bitte den
%% Befehl in \documentclass[manuscript, screen, review]{acmart}.
%%
%% Bei der Einreichung von druckfertigen Vorlagen oder bei TAPS �ndern Sie bitte den Befehl
%% in \documentclass[sigconf]{acmart} oder die f�r Ihre Ver�ffentlichung erforderliche Vorlage
%% .
%%
%%
\documentclass[manuscript,screen,review]{acmart}
%%
%% \BibTeX-Befehl zum Setzen des BibTeX-Logos in den Dokumenten
