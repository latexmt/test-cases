%Version 3.1 Dezember 2024
% Siehe Abschnitt 11 des Benutzerhandbuchs f�r die Versionshistorie.
%
%%%%%%%%%%%%%%%%%%%%%%%%%%%%%%%%%%%%%%%%%%%%%%%%%%%%%%%%%%%%%%%%%%%%%%
%%                                                                 %%
%% Bitte verwenden Sie nicht \input{...}, um andere tex-Dateien einzuf�gen.       %%
%% Reichen Sie Ihr LaTeX-Manuskript als ein einziges .tex-Dokument ein.              %%
%%                                                                 %%
%% Alle zus�tzlichen Abbildungen und Dateien sollten separat angeh�ngt werden             %%
%% und nicht in das \TeX\-Dokument selbst eingebettet werden.       %%
%%                                                                 %%
%%%%%%%%%%%%%%%%%%%%%%%%%%%%%%%%%%%%%%%%%%%%%%%%%%%%%%%%%%%%%%%%%%%%%

%%\documentclass[referee,sn-basic]{sn-jnl}% Die Option \u201ereferee\u201d ist f�r doppelten Zeilenabstand vorgesehen.

%%=======================================================%%
%% Um Zeilennummern am Rand zu drucken, verwenden Sie die Option lineno. %%
%%=======================================================%%

%%\documentclass[lineno,pdflatex,sn-basic]{sn-jnl}% Grundlegender Springer Nature-Referenzstil/Chemie-Referenzstil

%%=========================================================================================%%
%% Die Dokumentklasse ist standardm��ig auf pdflatex eingestellt. Sie k�nnen sie l�schen, wenn sie nicht geeignet ist.  %%
%%=========================================================================================%%

%%\documentclass[sn-basic]{sn-jnl}% Grundlegender Springer Nature-Referenzstil/Chemie-Referenzstil

%%Hinweis: 
