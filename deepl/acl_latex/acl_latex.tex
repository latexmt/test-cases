\documentclass[11pt]{article}

% �ndern Sie \u201ereview\u201d in \u201efinal\u201d, um die endg�ltige (manchmal auch als druckfertig bezeichnete) Version zu erstellen.
% �ndern Sie in \u201epreprint\u201d, um eine nicht anonymisierte Version mit Seitenzahlen zu erstellen.
\usepackage[review]{acl}

% Standardpaket enth�lt
\usepackage{times}
\usepackage{latexsym}

% F�r die korrekte Darstellung und Silbentrennung von W�rtern mit lateinischen Zeichen (auch in Bib-Dateien)
\usepackage[T1]{fontenc}
% F�r vietnamesische Zeichen
% \usepackage[T5]{fontenc}
% Weitere Zeichens�tze finden Sie unter https://www.latex-project.org/help/documentation/encguide.pdf

% Hier wird davon ausgegangen, dass Ihre Dateien als UTF8 kodiert sind
\usepackage[utf8]{inputenc}

% Dies ist nicht unbedingt erforderlich und kann auskommentiert werden,
% verbessert jedoch das Layout des Manuskripts
% und spart in der Regel etwas Platz.
\usepackage{microtype}

% Dies ist ebenfalls nicht unbedingt erforderlich und kann auskommentiert werden.
% Es verbessert jedoch die �sthetik des Textes in der
% Schreibmaschinenschrift.
\usepackage{inconsolata}

% Um Bilder in Ihr LaTeX-Dokument einzuf�gen, m�ssen Sie
% zus�tzliche Pakete hinzuf�gen.
\usepackage{graphicx}

% Wenn der Titel und die Autorenangaben nicht in den daf�r vorgesehenen Bereich passen, entfernen Sie die Auskommentierung der folgenden Zeile:
%
%\setlength\titlebox{<dim>}
%
% und setzen Sie <dim> auf einen Wert von mindestens 5 cm.

\title{Anweisungen f�r *ACL-Ver�ffentlichungen}

% Autorenangaben k�nnen in verschiedenen Stilen festgelegt werden:
% F�r mehrere Autoren derselben Institution:
% \author{Autor 1 \and ... \and Autor n \\
%         Adresszeile \\ ... \\ Adresszeile}
