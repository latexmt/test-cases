\documentclass[11pt]{article}

% Ändern Sie "review" in "final", um die endgültige (manchmal Kamera-fertige) Version zu erzeugen.
% Ändern Sie zu "preprint", um eine nicht-anonyme Version mit Seitenzahlen zu erstellen.
\usepackage[review]{acl}

% Standard-Pakete
\usepackage{times}
\usepackage{latexsym}

% Für korrekte Darstellung und Silbentrennung von Wörtern mit lateinischen Zeichen (auch in Bib-Dateien)
\usepackage[T1]{fontenc}
% Für vietnamesische Zeichen
% \usepackage[T5]{fontenc}
% Siehe https://www.latex-project.org/help/documentation/encguide.pdf für andere Zeichensätze

% Dies setzt voraus, dass Ihre Dateien als UTF8 codiert sind
\usepackage[utf8]{inputenc}

% Dies ist nicht zwingend notwendig und kann auskommentiert werden,
% aber es verbessert das Layout des Manuskripts
% und spart typischerweise Platz.
\usepackage{microtype}

% Ebenfalls nicht zwingend notwendig, kann aber auskommentiert werden.
% Es verbessert jedoch die Ästhetik der Schreibmaschinenschrift.
\usepackage{inconsolata}

% Das Einfügen von Bildern in Ihr LaTeX-Dokument erfordert zusätzliche Pakete
\usepackage{graphicx}

% Wenn Titel- und Autoreninformationen nicht in den vorgesehenen Bereich passen, heben Sie die folgende Auskommentierung auf
%
%\setlength\titlebox{<dim>}
%
% und setzen Sie <dim> auf einen Wert von 5cm oder mehr.

\title{Anweisungen für *ACL-Proceedings}

% Autoreninformationen können in verschiedenen Stilen angegeben werden:
% Für mehrere Autoren derselben Institution:
% \author{Autor 1 \and ... \and Autor n \\
%         Adresszeile \\ ... \\ Adresszeile}
% Falls die Namen nicht gut in eine Zeile passen, verwenden Sie:
%         Autor 1 \\ {\bf Autor 2} \\ ... \\ {\bf Autor n} \\
% Für Autoren verschiedener Institutionen:
% \author{Autor 1 \\ Adresszeile \\  ... \\ Adresszeile
%         \And  ... \And
%         Autor n \\ Adresszeile \\ ... \\ Adresszeile}
% Um eine separate „Reihe“ von Autoren zu beginnen, verwenden Sie \AND, wie in:
% \author{Autor 1 \\ Adresszeile \\  ... \\ Adresszeile
%         \AND
%         Autor 2 \\ Adresszeile \\ ... \\ Adresszeile \And
%         Autor 3 \\ Adresszeile \\ ... \\ Adresszeile}

\author{Erster Autor \\
  Zugehörigkeit / Adresszeile 1 \\
  Zugehörigkeit / Adresszeile 2 \\
  Zugehörigkeit / Adresszeile 3 \\
  \texttt{email@domain} \\\And
  Zweiter Autor \\
  Zugehörigkeit / Adresszeile 1 \\
  Zugehörigkeit / Adresszeile 2 \\
  Zugehörigkeit / Adresszeile 3 \\
  \texttt{email@domain} \\}

\begin{document}
\maketitle
\begin{abstract}
Dieses Dokument ist eine Ergänzung zu den allgemeinen Anweisungen für *ACL-Autoren. Es enthält Anweisungen zur Verwendung der \LaTeX{}-Stildateien für ACL-Konferenzen.
Das Dokument selbst entspricht seinen eigenen Spezifikationen und dient daher als Beispiel dafür, wie Ihr Manuskript aussehen sollte.
Diese Anweisungen sind sowohl für eingereichte Beiträge zur Begutachtung als auch für die endgültigen Versionen angenommener Arbeiten zu verwenden.
\end{abstract}

\section{Einleitung}

Diese Anweisungen sind für Autoren gedacht, die Beiträge für *ACL-Konferenzen mit \LaTeX{} einreichen. Sie sind nicht eigenständig. Alle Autoren müssen den allgemeinen Anweisungen für *ACL-Proceedings folgen,\footnote{\url{http://acl-org.github.io/ACLPUB/formatting.html}} und dieses Dokument enthält zusätzliche Anweisungen für die \LaTeX{}-Stildateien.

Die Vorlagen enthalten den \LaTeX{}-Quelltext dieses Dokuments (\texttt{acl\_latex.tex}),
die zum Formatieren verwendete \LaTeX{}-Stildatei (\texttt{acl.sty}),
einen ACL-Bibliographie-Stil (\texttt{acl\_natbib.bst}),
eine Beispielbibliographie (\texttt{custom.bib})
und die Bibliographie für die ACL Anthology (\texttt{anthology.bib}).

\section{Engines}

Zum Erstellen einer PDF-Datei wird pdf\LaTeX{} dringend empfohlen (statt ursprünglichem \LaTeX{} plus dvips+ps2pdf oder dvipdf).
Die Stildatei \texttt{acl.sty} kann auch mit
lua\LaTeX{} und
Xe\LaTeX{} verwendet werden, die sich besonders für Texte in nicht-lateinischen Schriften eignen.
Die Datei \texttt{acl\_lualatex.tex} in diesem Repository bietet
ein Beispiel für die Verwendung von \texttt{acl.sty} mit
lua\LaTeX{} oder
Xe\LaTeX{}.

\section{Präambel}

Die erste Zeile der Datei muss lauten:
\begin{quote}
\begin{verbatim}
\documentclass[11pt]{article}
\end{verbatim}
\end{quote}

Um die Stildatei in der Review-Version zu laden:
\begin{quote}
\begin{verbatim}
\usepackage[review]{acl}
\end{verbatim}
\end{quote}
Für die endgültige Version lassen Sie die Option \verb|review| weg:
\begin{quote}
\begin{verbatim}
\usepackage{acl}
\end{verbatim}
\end{quote}

Um Times Roman zu verwenden, fügen Sie Folgendes in die Präambel ein:
\begin{quote}
\begin{verbatim}
\usepackage{times}
\end{verbatim}
\end{quote}
(Alternativen wie txfonts oder newtx sind ebenfalls akzeptabel.)

Bitte sehen Sie den \LaTeX{}-Quelltext dieses Dokuments für Kommentare zu anderen nützlichen Paketen.

Titel und Autoren werden mit \verb|\title| und \verb|\author| gesetzt. Innerhalb der Autorenliste formatieren Sie mehrere Autoren mit \verb|\and|, \verb|\And| und \verb|\AND|; siehe den Quelltext für Beispiele.

Standardmäßig ist der Kasten mit Titel und Autoren auf mindestens 5 cm eingestellt. Wenn Sie mehr Platz benötigen, fügen Sie Folgendes in die Präambel ein:
\begin{quote}
\begin{verbatim}
\setlength\titlebox{<dim>}
\end{verbatim}
\end{quote}
wobei \verb|<dim>| durch eine Länge ersetzt wird. Setzen Sie diesen Wert nicht kleiner als 5 cm.

\section{Dokumenttext}

\subsection{Fußnoten}

Fußnoten werden mit dem Befehl \verb|\footnote| eingefügt.\footnote{Dies ist eine Fußnote.}

\subsection{Tabellen und Abbildungen}

Siehe Tabelle~\ref{tab:accents} für ein Beispiel einer Tabelle und ihrer Beschriftung.
\textbf{Ändern Sie nicht die Standardgrößen der Beschriftungen.}

\begin{table}
  \centering
  \begin{tabular}{lc}
    \hline
    \textbf{Befehl} & \textbf{Ausgabe} \\
    \hline
    \verb|{\"a}|     & {\"a}           \\
    \verb|{\^e}|     & {\^e}           \\
    \verb|{\`i}|     & {\`i}           \\
    \verb|{\.I}|     & {\.I}           \\
    \verb|{\o}|      & {\o}            \\
    \verb|{\'u}|     & {\'u}           \\
    \verb|{\aa}|     & {\aa}           \\\hline
  \end{tabular}
  \begin{tabular}{lc}
    \hline
    \textbf{Befehl} & \textbf{Ausgabe} \\
    \hline
    \verb|{\c c}|    & {\c c}          \\
    \verb|{\u g}|    & {\u g}          \\
    \verb|{\l}|      & {\l}            \\
    \verb|{\~n}|     & {\~n}           \\
    \verb|{\H o}|    & {\H o}          \\
    \verb|{\v r}|    & {\v r}          \\
    \verb|{\ss}|     & {\ss}           \\
    \hline
  \end{tabular}
  \caption{Beispielbefehle für Akzentzeichen, z.\,B. für Bib\TeX{}-Einträge.}
  \label{tab:accents}
\end{table}

Soweit möglich, sollten die Schriftarten in Abbildungen den Dokument-Schriften entsprechen. Siehe Abbildung~\ref{fig:experiments} für ein Beispiel einer Abbildung und Beschriftung.

Mit dem Paket \verb|graphicx| können Grafikdateien in einer figure-Umgebung an geeigneter Stelle eingefügt werden.
Das Paket \verb|graphicx| unterstützt verschiedene optionale Argumente zur Steuerung des Erscheinungsbilds der Abbildung.
Sie müssen es explizit in der Präambel einfügen, nach der
\verb|\documentclass|-Deklaration und vor \verb|\begin{document}|, mit
\verb|\usepackage{graphicx}|.

\begin{figure}[t]
  \includegraphics[width=\columnwidth]{example-image-golden}
  \caption{Eine Abbildung mit einer Beschriftung, die über mehrere Zeilen läuft.
    Das Beispielbild ist in der Regel über das Paket \texttt{mwe} verfügbar,
    ohne es in der Präambel erwähnen zu müssen.}
  \label{fig:experiments}
\end{figure}

\begin{figure*}[t]
  \includegraphics[width=0.48\linewidth]{example-image-a} \hfill
  \includegraphics[width=0.48\linewidth]{example-image-b}
  \caption {Ein minimales Arbeitsbeispiel, das zeigt,
    wie zwei Bilder nebeneinander platziert werden.}
\end{figure*}

\subsection{Hyperlinks}

Benutzer älterer \LaTeX{}-Versionen könnten auf den folgenden Fehler stoßen:
\begin{quote}
\verb|\pdfendlink| landete in einer anderen Verschachtelungsebene als \verb|\pdfstartlink|.
\end{quote}
Dies geschieht, wenn pdf\LaTeX{} verwendet wird und ein Zitat über eine Seitenbegrenzung hinausgeht. Die beste Lösung ist ein Upgrade auf \LaTeX{} 2018-12-01 oder neuer.

\subsection{Zitate}

% (Übersetzungen für die Tabelle und erklärenden Text)
\begin{table*}
  \centering
  \begin{tabular}{lll}
    \hline
    \textbf{Ausgabe}           & \textbf{natbib-Befehl} & \textbf{Nur-ACL-Befehl} \\
    \hline
    \citep{Gusfield:97}       & \verb|\citep|           &                           \\
    \citealp{Gusfield:97}     & \verb|\citealp|         &                           \\
    \citet{Gusfield:97}       & \verb|\citet|           &                           \\
    \citeyearpar{Gusfield:97} & \verb|\citeyearpar|     &                           \\
    \citeposs{Gusfield:97}    &                         & \verb|\citeposs|          \\
    \hline
  \end{tabular}
  \caption{\label{citation-guide}
    Von der Stildatei unterstützte Zitierbefehle.
    Der Stil basiert auf dem natbib-Paket und unterstützt alle natbib-Befehle.
    Außerdem unterstützt er ältere ACL-Befehle zur Kompatibilität.
  }
\end{table*}

Tabelle~\ref{citation-guide} zeigt die von den Stildateien unterstützte Syntax.
Wir empfehlen die Verwendung der natbib-Stile.
Mit dem Befehl \verb|\citet| erhalten Sie Zitate im Format „Autor (Jahr)“, z.\,B. \citet{Gusfield:97}.
Mit \verb|\citep| erhalten Sie „(Autor, Jahr)“-Zitate \citep{Gusfield:97}.
Mit \verb|\citealp| (Alternative ohne Klammern) erhalten Sie „Autor, Jahr“-Zitate, nützlich in Klammern (z.\,B. \citealp{Gusfield:97}).

Ein besitzanzeigendes Zitat kann mit \verb|\citeposs| erstellt werden. Dies ist kein Standard-natbib-Befehl und möglicherweise nicht mit anderen Stilen kompatibel.

\subsection{Literaturverzeichnis}

Die \LaTeX{}- und Bib\TeX{}-Stildateien folgen grob dem APA-Format.
Wenn Ihre Bib-Datei \texttt{custom.bib} heißt, erzeugen Sie den Literaturabschnitt mit:
\begin{quote}
\begin{verbatim}
\bibliography{custom}
\end{verbatim}
\end{quote}

Für Anthology und eigene .bib-Datei:
\begin{quote}
\begin{verbatim}
\bibliography{anthology,custom}
\end{verbatim}
\end{quote}

\subsection{Gleichungen}

Ein Beispiel für eine Gleichung:
\begin{equation}
  \label{eq:example}
  A = \pi r^2
\end{equation}

Dies ist ein Beispielverweis auf Gleichung~\ref{eq:example}.

\subsection{Anhänge}

Verwenden Sie \verb|\appendix| vor jedem Anhang, um die Nummerierung auf Buchstaben umzustellen. Siehe Anhang~\ref{sec:appendix} für ein Beispiel.

\section*{Einschränkungen}

Dieses Dokument behandelt nicht die Inhaltsanforderungen für ACL oder andere Veranstaltungen. Prüfen Sie die Autorenanweisungen für Informationen zu maximalen Seitenlängen, den erforderlichen Abschnitt „Einschränkungen“ usw.

\section*{Danksagungen}

Dieses Dokument wurde von Steven Bethard, Ryan Cotterell und Rui Yan
auf Basis früherer ACL- und NAACL-Anweisungen angepasst, einschließlich
ACL 2019 (Douwe Kiela und Ivan Vuli\'{c}),
NAACL 2019 (Stephanie Lukin und Alla Roskovskaya),
ACL 2018 (Shay Cohen, Kevin Gimpel und Wei Lu),
NAACL 2018 (Margaret Mitchell und Stephanie Lukin),
Bib\TeX{}-Vorschlägen für (NA)ACL 2017/2018 (Jason Eisner),
ACL 2017 (Dan Gildea und Min-Yen Kan),
NAACL 2017 (Margaret Mitchell),
ACL 2012 (Maggie Li und Michael White),
ACL 2010 (Jing-Shin Chang und Philipp Koehn),
ACL 2008 (Johanna D. Moore, Simone Teufel, James Allan und Sadaoki Furui),
ACL 2005 (Hwee Tou Ng und Kemal Oflazer),
ACL 2002 (Eugene Charniak und Dekang Lin),
sowie früheren ACL- und EACL-Formaten mehrerer Autoren, darunter
John Chen, Henry S. Thompson und Donald Walker.
Weitere Elemente stammen aus den Formatierungsanweisungen der \emph{International Joint Conference on Artificial Intelligence} und der \emph{Conference on Computer Vision and Pattern Recognition}.

\bibliography{custom}

\appendix

\section{Beispielanhang}
\label{sec:appendix}

Dies ist ein Anhang.

\end{document}
